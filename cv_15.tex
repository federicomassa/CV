%%%%%%%%%%%%%%%%%%%%%%%%%%%%%%%%%%%%%%%
% Wenneker Resume/CV
% LaTeX Template
% Version 1.1 (19/6/2016)
%
% This template has been downloaded from:
% http://www.LaTeXTemplates.com
%
% Original author:
% Frits Wenneker (http://www.howtotex.com) with extensive modifications by 
% Vel (vel@LaTeXTemplates.com)
%
% License:
% CC BY-NC-SA 3.0 (http://creativecommons.org/licenses/by-nc-sa/3.0/
%
%%%%%%%%%%%%%%%%%%%%%%%%%%%%%%%%%%%%%%

%----------------------------------------------------------------------------------------
%	PACKAGES AND OTHER DOCUMENT CONFIGURATIONS
%----------------------------------------------------------------------------------------

\documentclass[a4paper,12pt]{article} % Font and paper size, changed from memoir to article

%%%%%%%%%%%%%%%%%%%%%%%%%%%%%%%%%%%%%%%%%
% Wenneker Resume/CV
% Structure Specification File
% Version 1.1 (19/6/2016)
%
% This file has been downloaded from:
% http://www.LaTeXTemplates.com
%
% Original author:
% Frits Wenneker (http://www.howtotex.com) with extensive modifications by 
% Vel (vel@latextemplates.com)
%
% License:
% CC BY-NC-SA 3.0 (http://creativecommons.org/licenses/by-nc-sa/3.0/)
%
%%%%%%%%%%%%%%%%%%%%%%%%%%%%%%%%%%%%%%%%%

%----------------------------------------------------------------------------------------
%	PACKAGES AND OTHER DOCUMENT CONFIGURATIONS
%----------------------------------------------------------------------------------------

\usepackage{XCharter} % Use the Bitstream Charter font
\usepackage[utf8]{inputenc} % RequiRoyalBlue for inputting international characters
\usepackage[T1]{fontenc} % Output font encoding for international characters

\usepackage[top=1cm,left=.1cm,right=1cm,bottom=1cm]{geometry} % Modify margins

\usepackage{graphicx} % RequiRoyalBlue for figures

\usepackage{flowfram} % RequiRoyalBlue for the multi-column layout

\usepackage{url} % URLs

\usepackage[usenames,dvipsnames]{xcolor} % RequiRoyalBlue for custom colours

\usepackage{tikz} % RequiRoyalBlue for the horizontal rule

\usepackage{enumitem} % RequiRoyalBlue for modifying lists
\setlist{noitemsep,nolistsep} % Remove spacing within and around lists

\setlength{\columnsep}{\baselineskip} % Set the spacing between columns

% Define the left frame (sidebar)
\newflowframe{0.25\textwidth}{\textheight}{0pt}{0pt}[left]
\newlength{\LeftMainSep}
\setlength{\LeftMainSep}{0.25\textwidth}
\addtolength{\LeftMainSep}{1\columnsep}
 
% Small static frame for the vertical line
\newstaticframe{1.5pt}{\textheight}{\LeftMainSep}{0pt}
 
% Content of the static frame with the vertical line
\begin{staticcontents}{1}
\hfill
\tikz{\draw[loosely dotted,color=RoyalBlue,line width=1.5pt,yshift=0](0,0) -- (0,\textheight);}
\hfill\mbox{}
\end{staticcontents}
 
% Define the right frame (main body)
\addtolength{\LeftMainSep}{1.5pt}
\addtolength{\LeftMainSep}{1\columnsep}
\newflowframe{0.7\textwidth}{\textheight}{\LeftMainSep}{0pt}[main01]

\pagestyle{empty} % Disable all page numbering

\setlength{\parindent}{0pt} % Stop paragraph indentation

%----------------------------------------------------------------------------------------
%	NEW COMMANDS
%----------------------------------------------------------------------------------------

\newcommand{\userinformation}[1]{\renewcommand{\userinformation}{#1}} % Define a new command for the CV user's information that goes into the left column

\newcommand{\cvheading}[1]{{\Huge\bfseries\color{RoyalBlue} #1} \par\vspace{.6\baselineskip}} % New command for the CV heading
\newcommand{\cvsubheading}[1]{{\Large\bfseries #1} \bigbreak} % New command for the CV subheading

\newcommand{\Sep}{\vspace{1em}} % New command for the spacing between headings
\newcommand{\SmallSep}{\vspace{0.5em}} % New command for the spacing within headings

\newcommand{\aboutme}[2]{ % New command for the about me section
\textbf{\color{RoyalBlue} #1}~~#2\par\Sep
}
	
\newcommand{\CVSection}[1]{ % New command for the headings within sections
{\Large\textbf{#1}}\par
\SmallSep % Used for spacing
}

\newcommand{\CVItem}[2]{ % New command for the item descriptions
\textbf{\color{RoyalBlue} #1}\par
#2
\SmallSep % Used for spacing
}

\newcommand{\bluebullet}{\textcolor{RoyalBlue}{$\circ$}~~} % New command for the blue bullets
 % Include the file specifying document layout and packages
\usepackage{graphicx}
\usepackage{tabu}
%----------------------------------------------------------------------------------------
%	NAME AND CONTACT INFORMATION 
%----------------------------------------------------------------------------------------

\userinformation{ % Set the content that goes into the sidebar of each page
\begin{flushright}
% Comment out this figure block if you don't want a photo
\includegraphics[width=0.6\columnwidth]{photo.jpg}\\[\baselineskip] % Your photo
\small % Smaller font size
Federico Massa \\ % Your name
\url{fedemassa91@gmail.com}  \\
%\url{gmail.com} \\ % Your email address
%\url{www.johnsmith.com} \\ % Your URL
(+39) 347-7034248 \\ % Your phone number
\Sep % Some whitespace
\textbf{Address} \\
Via G. B. Pellizzi, 6 \\ % Address 1
56127 Pisa (PI) \\ % Address 2
Italy \\ % Address 3
\vfill % Whitespace under this block to push it up under the photo
\end{flushright}
}

%----------------------------------------------------------------------------------------

\begin{document}

\userinformation % Print your information in the left column

\framebreak % End of the first column

%----------------------------------------------------------------------------------------
%	HEADING
%----------------------------------------------------------------------------------------

\cvheading{Federico Massa} % Large heading - your name

\cvsubheading{Physicist} % Subheading - your occupation/specialization

%----------------------------------------------------------------------------------------
%	ABOUT ME
%----------------------------------------------------------------------------------------

\aboutme{About Me}{I was born in Cagliari in 1991, where I attended high school and received a Bachelor Degree in
Physics. Then I moved to Pisa to study for a Master's Degree in Experimental Particle Physics, where I am going to graduate with a thesis on the Monte Carlo simulation of a particle tracker. 
I am very passionate about the software aspect of my work, and I would like to apply 
my knowledge to the technology field.}

%----------------------------------------------------------------------------------------
%	EDUCATION
%----------------------------------------------------------------------------------------

\CVSection{Education}

%------------------------------------------------

\CVItem{2013 - 2016, Universit\`a di Pisa}{MSc in Physics of Fundamental Interactions (Graduation in Sep. 2016).}

%------------------------------------------------
%\CVItem{Jun 2016: Alghero}{Attendance to the XIII Seminar on Nuclear, Subnuclear and Applied Physics}

%\CVItem{Aug 2014: Göttingen}{Attendance to the HASCO Summer School on Hadron Colliders}

\CVItem{2010 - 2013, Universit\`a degli Studi di Cagliari}{BSc in Physics - 110/110 cum laude.}

%------------------------------------------------

\CVItem{2005 - 2010, Liceo Scientifico Pitagora - Selargius (CA)}{High School Diploma (Diploma di Maturit\`a Scientifica) - 100/100 cum laude.}

%\Sep % Extra whitespace after the end of a major section

%----------------------------------------------------------------------------------------
%	EXPERIENCE
%----------------------------------------------------------------------------------------

%\CVSection{Experience}

%------------------------------------------------

During the BSc I received strong foundations on Mathematics, Classical Physics, basic experimental
techniques, statistics and electronics (FPGA design with Quartus II, using Verilog for hardware
description). My BSc thesis was focused on the phenomenological simulation of a particle detector
to assess the feasibility of an experiment with the goal of measuring the sensibility to
new physics processes.\\

During the MSc, instead, I focused on several experimental aspects regarding High Energy Physics,
which included a more deep study on Statistical Data Analysis, Electronics and Monte Carlo simulations. I also attended a course on Image Processing and Computer Vision. My MSc
thesis focused on the Monte Carlo simulation and the analysis of performance of different proposed 
configurations of the Inner Tracker (ITk), which is designed to track charged particles produced during the proton-proton collisions, and will replace the current Inner Detector of the ATLAS experiment during the high-luminosity phase of the Large Hadron Collider (Geneva), in 2026. \\

During the MSc, I attended the HASCO Summer School on Hadron Colliders in Göttingen,  
 and the XIII Seminar on Nuclear, Subnuclear and Applied Physics in Alghero (see attachments). The latter was
 centered on the study of the Geant4 software package (a framework to simulate the
 interaction of particles with matter) and GPU parallel programming. \\
%------------------------------------------------

%------------------------------------------------

\Sep % Extra whitespace after the end of a major section

%----------------------------------------------------------------------------------------
%	COMMUNICATION SKILLS
%----------------------------------------------------------------------------------------
\CVSection{Languages}
\begin{table}[h!]
\centering
\resizebox{.71\textwidth}{!}{\tabulinesep=1.2mm{\begin{tabu}{| l | c | c | c | c | c |}
\cline{2-6}
\multicolumn{1}{c}{ } & \multicolumn{2}{|c|}{\textbf{Understanding}} & \multicolumn{2}{c|}{\textbf{Speaking}} & \textbf{Writing} \\ \cline{2-6}
\multicolumn{1}{c}{ } & \multicolumn{1}{|c|}{Listening} & Reading & Spoken interaction & Spoken production & \\ \hline
English & C1 & C1 & C1 & C1 & C1 \\ \hline
Spanish & B2 & B2 & B2 & B2 & B2 \\ \hline
\end{tabu}}}
\end{table}

The corresponding certificates can be found in the attachments.

\clearpage
\userinformation
\framebreak

\CVSection{Communication Skills}

%------------------------------------------------

\CVItem{2015-2016, \textit{Oral Presentations}}{During my Master's Degree thesis,
 I presented my work a large number of times to different ATLAS research groups: 
 \begin{itemize}
 \item ATLAS Pisa;
 \item ITk Simulation \& Performance;
 \item Upgrade Tracking;
 \item Physics Upgrade;
 \item ITk Layout Taskforce.
 \end{itemize}
 }

%------------------------------------------------
\Sep
I really enjoy, and I have a many years' experience in teaching Physics and Mathematics to students from Middle School to BSc.\\

%------------------------------------------------

%\Sep % Extra whitespace after the end of a major section

%----------------------------------------------------------------------------------------
%	SKILLS
%----------------------------------------------------------------------------------------

\CVSection{Software Skills}

%------------------------------------------------

\CVItem{Programming}
{\begin{tabular}{p{0.2\textwidth} p{0.2\textwidth} p{0.2\textwidth}}
\bluebullet C++ &  \bluebullet Java & \bluebullet Unix shell\\
\bluebullet Python &  \bluebullet Mathematica & \bluebullet Matlab\\
\bluebullet Fortran 90 & \bluebullet Visual Basic & \\
\end{tabular}}

%------------------------------------------------

\CVItem{Operative systems}
{\begin{tabular}{p{0.2\textwidth} p{0.2\textwidth} p{0.2\textwidth}}
 \bluebullet Windows &  \bluebullet Linux & \bluebullet Android\\
\end{tabular}}

%-----------------------------------------------

\CVItem{Computer software}
{\begin{tabular}{p{0.2\textwidth} p{0.2\textwidth} p{0.2\textwidth}}
 \bluebullet Internet browsers & \bluebullet Microsoft Office & \bluebullet LaTeX \\
 \bluebullet ROOT Framework & \bluebullet Geant4 & \bluebullet Gnuplot \\
 \bluebullet Qt Creator & \bluebullet Eclipse & \bluebullet Android Studio \\
\end{tabular}}



%------------------------------------------------

\Sep % Extra whitespace after the end of a major section

%----------------------------------------------------------------------------------------
%	NEW PAGE DELIMITER
%	Place this block wherever you would like the content of your CV to go onto the next page
%----------------------------------------------------------------------------------------

%----------------------------------------------------------------------------------------
%	AWARDS
%----------------------------------------------------------------------------------------

%\CVSection{Awards}

%------------------------------------------------

%\CVItem{2010, \textit{Postgraduate Scholarship}, Cornell University}{Awarded to the top student in their final year of a Bachelors degree.}

%------------------------------------------------

%\Sep % Extra whitespace after the end of a major section

%----------------------------------------------------------------------------------------
%	INTERESTS
%----------------------------------------------------------------------------------------

\CVSection{Interests}

%------------------------------------------------

\CVItem{Professional}{Data analysis, company profiling, risk analysis, economics, web design, web app creation, software design, marketing}

%------------------------------------------------

\CVItem{Personal}{Piano, chess, cooking, dancing, running}

%------------------------------------------------

\Sep % Extra whitespace after the end of a major section

%----------------------------------------------------------------------------------------

\end{document}

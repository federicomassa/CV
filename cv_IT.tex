%%%%%%%%%%%%%%%%%%%%%%%%%%%%%%%%%%%%%%%
% Wenneker Resume/CV
% LaTeX Template
% Version 1.1 (19/6/2016)
%
% This template has been downloaded from:
% http://www.LaTeXTemplates.com
%
% Original author:
% Frits Wenneker (http://www.howtotex.com) with extensive modifications by 
% Vel (vel@LaTeXTemplates.com)
%
% License:
% CC BY-NC-SA 3.0 (http://creativecommons.org/licenses/by-nc-sa/3.0/
%
%%%%%%%%%%%%%%%%%%%%%%%%%%%%%%%%%%%%%%

%----------------------------------------------------------------------------------------
%	PACKAGES AND OTHER DOCUMENT CONFIGURATIONS
%----------------------------------------------------------------------------------------

\documentclass[a4paper,12pt]{article} % Font and paper size, changed from memoir to article

\input{structure.tex} % Include the file specifying document layout and packages
\usepackage{graphicx}
\usepackage[italian]{babel}
\usepackage{tabu}
%----------------------------------------------------------------------------------------
%	NAME AND CONTACT INFORMATION 
%----------------------------------------------------------------------------------------

\userinformation{ % Set the content that goes into the sidebar of each page
\begin{flushright}
% Comment out this figure block if you don't want a photo
\includegraphics[width=0.6\columnwidth]{fede_fototessera.jpg}\\[\baselineskip] % Your photo
\small % Smaller font size
Federico Massa \\ % Your name
\url{fedemassa91@gmail.com}  \\
%\url{gmail.com} \\ % Your email address
%\url{www.johnsmith.com} \\ % Your URL
(+39) 347-7034248 \\ % Your phone number
\Sep % Some whitespace
\textbf{Indirizzo} \\
Via G. B. Pellizzi, 6 \\ % Address 1
56127 Pisa (PI) \\ % Address 2
Italy \\ % Address 3
\vfill % Whitespace under this block to push it up under the photo
\end{flushright}
}

%----------------------------------------------------------------------------------------

\begin{document}

\userinformation % Print your information in the left column

\framebreak % End of the first column

%----------------------------------------------------------------------------------------
%	HEADING
%----------------------------------------------------------------------------------------

\cvheading{Federico Massa} % Large heading - your name

\cvsubheading{Fisico} % Subheading - your occupation/specialization

%----------------------------------------------------------------------------------------
%	ABOUT ME
%----------------------------------------------------------------------------------------

%----------------------------------------------------------------------------------------
%	EDUCATION
%----------------------------------------------------------------------------------------

\CVSection{Istruzione}

%------------------------------------------------
\CVItem{2016 - 2017, Formazione post-lauream, Universit\`a di Pisa}{Corsi di Robotica (Prof. A. Bicchi) e Teoria Dei Sistemi (Dott. M. Bianchi) presso la Facolt\`a di Ingegneria.}


\CVItem{2013 - 2016, Universit\`a di Pisa}{Laurea Magistrale in Fisica (curriculum Interazioni Fondamentali)\\ - 110/110 e lode.\\
- Titolo della tesi: ``Tracking performances of the ATLAS detector for the\\ HL-LHC and impact on 
the H $\rightarrow 4\mu$ channel''.}

%------------------------------------------------


\CVItem{2010 - 2013, Universit\`a degli Studi di Cagliari}{Laurea Triennale in Fisica - 110/110 e lode.\\
- Titolo della tesi: ``Impact of physics beyond the Standard Model on the diffusion of neutrinos on polarized electrons''.}

%------------------------------------------------

\CVItem{2010 - 2015}{Vincitore degli Assegni di Merito della Regione Sardegna}

\CVItem{2005 - 2010, Liceo Scientifico Pitagora - Selargius (CA)}{Diploma di Maturit\`a Scientifica - 100/100 e lode.}



\CVSection{Esperienze}

\CVItem{1 Novembre 2016 - in corso: Pisa}{Contratto di collaborazione occasionale presso il Centro Ricerche E. Piaggio (Università di Pisa)}

\CVItem{Febbraio 2016 - CERN (Ginevra)}{Stage al CERN per 
collaborazione con gruppo di ricerca ITk}

\CVItem{5 - 10 Giugno 2016: Alghero}{Partecipazione al workshop \textit{XIII Seminar on Nuclear, Subnuclear and Applied Physics}}

\CVItem{20 Luglio - 1 Agosto 2014: Göttingen}{Partecipazione alla \textit{HASCO Summer School on Hadron Colliders}}

\Sep % Extra whitespace after the end of a major section
\CVItem{Presso il Centro E. Piaggio: }{
Il mio lavoro presso il Centro E. Piaggio ha riguardato principalmente due tematiche di ricerca. La prima \`e relativa allo studio e alla simulazione di tecniche distribuite per il coordinamento di veicoli autonomi. In particolare, il lavoro \`e stato incentrato sulla definizione di algoritmi di controllo per automobili che viaggiano in \textit{platooning} in un contesto autostradale. L'obiettivo \`e stato raggiunto attraverso uno schema di interazione lineare basato su informazioni localmente disponibili ad ogni veicolo tramite l'uso di sensori di bordo. Partendo da questo ambito applicativo, come seconda tematica di ricerca, \`e stato sviluppato un framework generale per la risoluzione  del problema della \mbox{\textit{intrusion detection}} per sistemi multi-robot cooperanti. Lo scopo \`e stato quello di sviluppare un algoritmo distribuito che permettesse ad ogni veicolo di verificare se il comportamento osservato nei vicini fosse consistente con un certo set di regole di cooperazione, utilizzando dapprima solo i dati provenienti dai suoi sensori. Alla fine di questo processo di \textit{monitoring} viene elaborata una reputazione per il veicolo osservato. Eventuali limitazioni }
%----------------------------------------------------------------------------------------
%	EXPERIENCE
%----------------------------------------------------------------------------------------

%\CVSection{Experience}

%------------------------------------------------

%------------------------------------------------

%------------------------------------------------
\clearpage
\userinformation
\framebreak
dovute alla parziale visibilit\`a dell'osservatore possono essere risolte tramite la comunicazione con gli altri agenti. Questo problema \`e stato affrontato sotto due ipotesi di lavoro distinte. Nel primo caso si assume noto il modello dinamico degli agenti, che, nello specifico, consiste in un modello ibrido in cui l'agente pu\`o trovarsi in vari stati discreti, ognuno caratterizzato dal proprio controllore. Ci\`o che viene verificato in questo caso \`e che l'agente segua le corrette regole di transizione tra gli stati discreti. Il problema \`e complicato dalla presenza di controllori (come quello che regola il platooning) il cui output dipende dagli stati continui dei vicini. Nel secondo caso, invece, si mira a verificare che il comportamento dell'agente osservato rispetti delle regole sociali (nel contesto autostradale rappresentate dal codice stradale) descritte in maniera astratta. In questo caso si assume solo di conoscere alcuni comportamenti elementari dell'agente (nel caso autostradale alcuni esempi sono i cambi di corsia o i sorpassi), ma non si ha alcun dettaglio
sul modello dinamico (al pi\`u si assumono dei limiti di plausibilit\`a sul come viene effettuata una certa manovra). Anche in questo caso la reputazione pu\`o essere elaborata avvalendosi delle informazioni comunicate da altri agenti.

\Sep % Extra whitespace after the end of a major section

%----------------------------------------------------------------------------------------
%	COMMUNICATION SKILLS
%----------------------------------------------------------------------------------------
\CVSection{Competenze linguistiche}
\begin{table}[h!]
\centering
\resizebox{.71\textwidth}{!}{\tabulinesep=1.2mm{\begin{tabu}{| l | c | c | c | c | c |}
\cline{2-6}
\multicolumn{1}{c}{ } & \multicolumn{2}{|c|}{\textbf{Comprensione}} & \multicolumn{2}{c|}{\textbf{Orale}} & \textbf{Produzione scritta} \\ \cline{2-6}
\multicolumn{1}{c}{ } & \multicolumn{1}{|c|}{Ascolto} & Lettura & Interazione & Produzione & \\ \hline
Inglese & C1 & C1 & C1 & C1 & C1 \\ \hline
Spagnolo & B2 & B2 & B2 & B2 & B2 \\ \hline
\end{tabu}}}
\end{table}

\Sep 

\CVSection{Abilità di comunicazione}

%------------------------------------------------

\CVItem{2015-2016, Presentazioni}{Durante il periodo di tesi della Laurea Magistrale, ho presentato il mio lavoro in numerose occasioni a diversi gruppi di ricerca dell'esperimento ATLAS:
 \begin{itemize}
 \item ATLAS Pisa;
 \item ITk Simulation \& Performance;
 \item Upgrade Tracking;
 \item Physics Upgrade;
 \item ITk Layout Taskforce.
 \end{itemize}
 }

%------------------------------------------------
\Sep
Ho una forte propensione ed un'esperienza pluriennale nell'assistenza allo studio della
Fisica e della Matematica a studenti dalla scuola media fino alla Laurea Triennale. \\

%------------------------------------------------

%\Sep % Extra whitespace after the end of a major section

%----------------------------------------------------------------------------------------
%	SKILLS
%----------------------------------------------------------------------------------------

\CVSection{Abilità specifiche}
\begin{itemize}
\item Modellizzazione e simulazione di sistemi robotici distribuiti e cooperanti;
\item Tecniche di coordinamento di veicoli;
\item Analisi matematica;
\item Meccanica classica e quantistica;
\item Elettromagnetismo;
\item Analisi statistica dei dati;
\item Tecniche sperimentali;
\item Fondamenti di elettronica digitale e analogica;
\item Simulazioni Monte Carlo;
\item Image processing e Computer Vision.
\end{itemize}

\clearpage
\userinformation
\framebreak

\Sep 

\CVSection{Abilità informatiche}

%------------------------------------------------

\CVItem{Programmazione}
{\begin{tabular}{p{0.2\textwidth} p{0.2\textwidth} p{0.2\textwidth}}
\bluebullet C++ &  \bluebullet Java & \bluebullet Shell Unix\\
\bluebullet Python &  \bluebullet Mathematica & \bluebullet MATLAB\\
\bluebullet Fortran 90 & \bluebullet Visual Basic & \bluebullet C\\
\bluebullet LaTeX
\end{tabular}}

%------------------------------------------------

\CVItem{Sistemi operativi}
{\begin{tabular}{p{0.2\textwidth} p{0.2\textwidth} p{0.2\textwidth}}
 \bluebullet Windows &  \bluebullet Unix/Linux & \bluebullet Android\\
\end{tabular}}

%-----------------------------------------------

\CVItem{Software utilizzati}
{\begin{tabular}{p{0.2\textwidth} p{0.2\textwidth} p{0.2\textwidth}}
 \bluebullet ROOT framework & \bluebullet Geant4 framework & \bluebullet Gnuplot \\
 \bluebullet Android Studio & \bluebullet Qt Creator & \bluebullet Eclipse \\
 \bluebullet Matlab e SimuLink & \bluebullet Texmaker (editor) & \bluebullet Mathematica \\
 \bluebullet{Unreal Engine 4} 
\end{tabular}}

Tra i vari linguaggi, ho una lunga esperienza di programmazione in C++ (in particolare per la simulazione del rivelatore di particelle effettuata durante il periodo di tesi e per la simulazione di sistemi robotici cooperanti) e Java.\\

%------------------------------------------------

\Sep % Extra whitespace after the end of a major section

%----------------------------------------------------------------------------------------
%	NEW PAGE DELIMITER
%	Place this block wherever you would like the content of your CV to go onto the next page
%----------------------------------------------------------------------------------------

%----------------------------------------------------------------------------------------
%	AWARDS
%----------------------------------------------------------------------------------------

%\CVSection{Awards}

%------------------------------------------------

%\CVItem{2010, \textit{Postgraduate Scholarship}, Cornell University}{Awarded to the top student in their final year of a Bachelors degree.}

%------------------------------------------------

%\Sep % Extra whitespace after the end of a major section

%----------------------------------------------------------------------------------------
%	INTERESTS
%----------------------------------------------------------------------------------------

\CVSection{Interessi}

%------------------------------------------------

\CVItem{Professionali}{Sviluppo Software per dispositivi mobili e desktop, Intelligenza Artificiale, Simulazione numerica.}

%------------------------------------------------

\CVItem{Personali}{Programmazione Arduino, Musica,  
mountain-bike, video-games.}

%------------------------------------------------

\Sep % Extra whitespace after the end of a major section

%----------------------------------------------------------------------------------------
\CVSection{Allegati}

\textit{Certificato di Laurea Triennale con esami}\\
\textit{Certificato di Laurea Magistrale con esami}\\
\textit{Certificato di lingua inglese}\\
\textit{Certificato di lingua spagnola}\\
\textit{Certificato di superamento del test finale al workshop ad Alghero}\\
\textit{Fotocopia del Documento di Identit\`a}\\

\bigskip
\textit{Autorizzo il trattamento dei miei dati personali, ai sensi del D.lgs. 196 del 30 giugno 2003.}\\

\end{document}
